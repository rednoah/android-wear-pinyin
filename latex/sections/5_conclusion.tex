\chapter{Limitation and Future Work}
\label{c:limitation and future work}


There are several limitations in both implementation and evaluation of our keyboard prototypes.

Firstly, we used a single device with a 1.2" screen for the entire study. The results may vary on smaller devices. More research is needed on how these techniques fare on different smartwatches of various form factors.

Secondly, each participant was only able to use each keyboard for about 20 minutes which puts unfamiliar input methods such as Pinyin Syllables at a distinct disadvantage. In addition, some of our participants were non-native speakers and it turns out that they had trouble memorizing and transcribing each sentence correctly due to the increased cognitive load of using an unfamiliar keyboard.


\chapter{Conclusion}
\label{c:conclusion}


In this paper, we compare three keyboard layouts for Chinese text entry on smartwatches. We implement a standard QWERTY keyboard and use it as a baseline to evaluate two novel adaptive keyboard layouts specifically designed for smartwatches that use unique features of the Chinese language model and Pinyin romanization system to allow for larger keys or keys that are mapped to multiple Latin letters reducing errors and number of keystrokes required to enter a given Chinese character.

The results of our user study show that 8 out of 15 participants prefer one of our novel input methods to the standard QWERTY keyboard despite QWERTY performing slightly better in terms of Chinese characters per minute (CCPM) due to the familiarity of the keyboard layout.
